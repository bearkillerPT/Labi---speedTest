
\documentclass{report}
\usepackage{pgfplots}
\usepackage[T1]{fontenc} % Fontes T1
\usepackage[utf8]{inputenc} % Input UTF8
\usepackage[nottoc]{tocbibind}
\usepackage{csquotes}
\usepackage[portuguese]{babel} %Usar língua portuguesa
\usepackage{blindtext} % Gerar texto automaticamentez	
\usepackage[printonlyused]{acronym}
\usepackage{hyperref} % para autoref
\usepackage{graphicx}


\pgfplotsset{
  compat=newest,
  xlabel near ticks,
  ylabel near ticks
}
\begin{document}
%%
% Definições
%

\def\titulo{Trabalho de aprofundamento AP2}
\def\data{9 de Abril de 2019}
\def\autores{André Patacas, Gil Teixeira}
\def\autorescontactos{(93357) andrepatacas@ua.pt, (88194) gilteixeira@ua.pt}
\def\departamento{DETI}
\def\logotipo{ua.pdf}
%
%%%%%% CAPA %%%%%%
%
\begin{titlepage}

\begin{center}
%
\vspace*{50mm}
%
{\Huge \titulo}\\ 
%
\vspace{10mm}
%
{\LARGE \autores}\\ 
%
\vspace{30mm}
%
\begin{figure}[h]
\center
\includegraphics{\logotipo}
\end{figure}
%
\vspace{30mm}
\end{center}
%
\begin{flushright}

\end{flushright}
\end{titlepage}

%%  Página de Título %%
\title{%
{\Huge\textbf{Aplicação para o cálculo de Largura de Banda e de latência}}\\
{\Large \departamento}
}
%
\author{%
    \autores \\
    \autorescontactos 
}

%
\date{\data}
%
\maketitle

\pagenumbering{roman}





\tableofcontents
% \listoftables     % descomentar se necessário
% \listoffigures    % descomentar se necessário


%%%%%%%%%%%%%%%%%%%%%%%%%%%%%%%
\clearpage
\pagenumbering{arabic}

%%%%%%%%%%%%%%%%%%%%%%%%%%%%%%%%
%%%%%% RESUMO %%%%%%
\begin{abstract}
Este relatório serve para descrever uma ferramenta desenvolvida para calcular a largura de banda e a latência da máquina, onde a aplicação se encontra a correr, a um determinado servidor ou a um conjunto, de cardinalidade especificável, de servidores de um país, sendo este também especificável.

\end{abstract}


\chapter{Introdução}
\label{chap.introducao}

A aplicação foi desenvolvida em python3 no âmbito da disciplina de Laboratórios de Informática, no ano letivo 2018/2019. A adicionar às especificações básicas pedidas, segundo o guião sobre regras do segundo trabalho de aprofundamento, construi-se ainda suporte para pydocs para haver uma explicação mais detalhada sobre cada método e classe no nosso projeto. O programa foi escrito com base em test driven development (\autoref{chap.metodologia}) e como tal os testes unitários e funcionais foram criados primeiro, seguidos por um esqueleto do programa e finalmente por vários updates a ambos ({chap.resultados}) para chegar ao estado em que a aplicação se encontra de momento (\autoref{chap.analise}). Finalmente são tiradas as conclusões sobre os aspetos positivos e, potencialmente, negativos desta solução em concreto (\autoref{chap.conclusao})


\chapter{Metodologia}
\label{chap.metodologia}

\begin{enumerate}
	\item Criar o esqueleto do programa que é agora o inicializador da classe (labi02) se esta for chamada diretamente;
	\item Criar o ficheiro test\_labi\_02 como um teste que, apenas se a construção da aplicação for robusta e exatamente como especificada, passa.
	\item Criar o programa labi\_02 e definir as funções com os argumentos de entrada e cada uma com uma descrição detalhada, disponivel nos pydocs, dos aspetos funcionais de cada função.
	\item Ajustar os métodos de forma a que a aplicação passa todos os testes impostos no teste criado.
	\item Testar o programa manualmente e/ou com testes funcionais.
	\item Corrigir eventuais erros.
	\item Iterar o processo de debugging e correção de erros.
\end{enumerate}


\chapter{Aplicação de Speed Test}
\label{chap.Aplicação de Speed Test}
\section{Labi\_02}
\hspace{5mm}Esta é a aplicação que foi desenvolvida e que pode ser utilizada diretamente de acordo com o usage demonstrado ao correr a aplicação sem argumentos. Toda a descrição feita neste relatório remete na mesma para a documentação, esta criada a quando do desenvolvimento da aplicação.

\subsection{calc\_download}
\hspace{5mm}Cálculo da largura de banda:\\
Este método pede, inicialmente, para fazer um download de 100 megabytes ao target server dentro de 10 segundos. Depois verifica que não há mais data para ser recebida do \textit{target\_server} e finalmente calcula o time download 1mb que a máquina demora a fazer download de 1 megabyte.\\ 
\hspace{5mm}\textbf{Argumentos}: \textit{target\_server}(dicionário com informação sobre o target server). 
\hspace{5mm}\textbf{Retorna}: float(1/\textit{time\_download\_1mb})

\subsection{calc\_latency}
\hspace{5mm}Cálculo da latência:\\ 
Este método tenta trocar dez comandos PING-PONG com o targe-server e calcula o tempo médio em milisegundos entre estas trocas.\\
\hspace{5mm}\textbf{Argumentos}: target server(dicionário com informação sobre o target server). 
\hspace{5mm}\textbf{Retorna}: int(tempo entre trocas em ms).

\subsection{country\_test}
\hspace{5mm}Este método serve para calcular o tempo de download e latency a um servidor random do país passado como argumento:\\ 
\hspace{5mm}\textbf{Argumentos}: target country(str).\\
\hspace{5mm}\textbf{Retorna}: objeto \textit{SpeedTestResult} com as informações relativas aos resultados do teste.

\chapter{Resultados}
\label{chap:resultados}
\chapter{Análise}
\label{chap:analise}
\chapter{Conclusão}
\label{chap:conclusao}
%%%%%%%%%%%%%%%%%%%%%%%%%%%%%%%%%



%%%%%%%%%%%%%%%%%%%%%%%%%%%%%%%%%


\end{document}
