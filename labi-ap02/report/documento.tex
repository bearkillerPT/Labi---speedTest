\documentclass{report}
\usepackage[T1]{fontenc} % Fontes T1
\usepackage[utf8]{inputenc} % Input UTF8
\usepackage[backend=biber, style=ieee]{biblatex} % para usar bibliografia
\usepackage{csquotes}
\usepackage[portuguese]{babel} %Usar língua portuguesa
\usepackage{blindtext} % Gerar texto automaticamente
\usepackage[printonlyused]{acronym}
\usepackage{hyperref} % para autoref
\usepackage{graphicx}

\bibliography{bibliografia}


\begin{document}
%%
% Definições
%
\def\titulo{Trabalho de aprofundamento AP2}
\def\data{9 de Abril de 2019}
\def\autores{André Patacas, Gil Teixeira}
\def\autorescontactos{(93357) andrepatacas@ua.pt, (88194) gilteixeira@ua.pt}
\def\departamento{DETI}
\def\logotipo{ua.pdf}
%
%%%%%% CAPA %%%%%%
%
\begin{titlepage}

\begin{center}
%
\vspace*{50mm}
%
{\Huge \titulo}\\ 
%
%
\vspace{10mm}
%
{\LARGE \autores}\\ 
%
\vspace{30mm}
%
\begin{figure}[h]
\center
\includegraphics{\logotipo}
\end{figure}
%
\vspace{30mm}
\end{center}
%
\begin{flushright}
\end{flushright}
\end{titlepage}

%%  Página de Título %%
\title{%
{\Huge\textbf{Aplicação para o cálculo de Largura de Banda e de latência}}\\
{\Large \departamento}
}
%
\author{%
    \autores \\
    \autorescontactos
}
%
\date{\data}
%
\maketitle

\pagenumbering{roman}

%%%%%% RESUMO %%%%%%
\begin{abstract}
Este relatório serve para descrever uma ferramenta desenvolvida para calcular a largura de banda e a latência da máquina, onde a aplicação se encontra a correr, a um determinado servidor ou a um conjunto, de cardinalidade especificável, de servidores de um país, sendo este também especificável.

\end{abstract}

%%%%%% Agradecimentos %%%%%%
% Segundo glisc deveria aparecer após conclusão...
\renewcommand{\abstractname}{Agradecimentos}
\begin{abstract}
Eventuais agradecimentos.
Comentar bloco caso não existam agradecimentos a fazer.
\end{abstract}


\tableofcontents
% \listoftables     % descomentar se necessário
% \listoffigures    % descomentar se necessário


%%%%%%%%%%%%%%%%%%%%%%%%%%%%%%%
\clearpage
\pagenumbering{arabic}

%%%%%%%%%%%%%%%%%%%%%%%%%%%%%%%%
\chapter{Introdução}
\label{chap.introducao}

A aplicação foi desenvolvida em python3 no âmbito da disciplina de Laboratórios de Informática, no ano letivo 2018/2019. A adicionar às especificações básicas pedidas, segundo o guião sobre regras do segundo trabalho de aprofundamento, construi-se ainda suporte para pydocs para haver uma explicação mais detalhada sobre cada método e classe no nosso projeto. O programa foi escrito com base em test driven development (\autoref{chap.metodologia}) e como tal os testes unitários e funcionais foram criados primeiro, seguidos por um esqueleto do programa e finalmente por vários updates a ambos ({chap.resultados}) para chegar ao estado em que a aplicação se encontra de momento (\autoref{chap.analise}). Finalmente são tiradas as conclusões sobre os aspetos positivos e, potencialmente, negativos desta solução em concreto (\autoref{chap.conclusao})


\chapter{Metodologia}
\label{chap.metodologia}

\begin{enumerate}
	\item Criar o esqueleto do programa que é agora o inicializador da classe se esta for chamada como main. (if \_\_name\_\_ == '\_\_main\_\_':);
	\item Criar o ficheiro test\_labi\_02 como um teste que, apenas se a construção da aplicação for robusta e exatamente como especificada, passa.
	\item Criar o programa labi\_02 e definir as funções com os argumentos de entrada e cada uma com uma descrição detalhada, disponivel nos pydocs, dos aspetos funcionais de cada função.
	\item Ajustar os métodos de forma a que a aplicação passa todos os testes impostos no teste criado.
	\item Testar o programa manualmente e/ou com testes funcionais.
	\item Corrigir eventuais erros.
	\item Iterar o processo de debugging e correção de erros.
\end{enumerate}

\section{Exemplos}

\subsection{Utilização de acrónimos}
Esta é a primeira invocação do acrónimo \ac{ua}.
E esta é a segunda: \ac{ua}.

Outras duas referências a \ac{miect}
e \ac{miect}.

\subsection{Referências bibliográficas}
Informação relativa à estrutura formal de um relatório pode ser obtida
na página do \ac{glisc}\cite{glisc}.

\chapter{Resultados}
\label{chap.resultados}
Descreve os resultados obtidos.

\chapter{Análise}
\label{chap.analise}
Analisa os resultados.

\chapter{Conclusões}
\label{chap.conclusao}
Apresenta conclusões.

\chapter*{Contribuições dos autores}
Resumir aqui o que cada autor fez no trabalho.
Usar abreviaturas para identificar os autores,
por exemplo AS para António Silva.
No fim indicar a percentagem de contribuição de cada autor.

%%%%%%%%%%%%%%%%%%%%%%%%%%%%%%%%%
\chapter*{Acrónimos}
\begin{acronym}
\acro{ua}[UA]{Universidade de Aveiro}
\acro{miect}[MIECT]{Mestrado Integrado em Engenharia de Computadores e Telemática}
\acro{lei}[LEI]{Licenciatura em Engenharia Informática}
\acro{glisc}[GLISC]{Grey Literature International Steering Committee}
\end{acronym}


%%%%%%%%%%%%%%%%%%%%%%%%%%%%%%%%%
\printbibliography

\end{document}
